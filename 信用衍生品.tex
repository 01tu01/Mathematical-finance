\section{信用衍生品}
\begin{enumerate}
    \item 假设A是信用保护买方,B是信用保护卖方,A以每年500基点购买C公司面值100万美元的2年期CDS,假设以现金交割,在如下两种情况下,分析B公司的收益情况:
    \begin{enumerate}[label=(\arabic*)]
        \item 两年末C违约。
        \item 两年末C未违约。
    \end{enumerate}
    \sol
    \begin{enumerate}[label=(\arabic*)]
        \item 若C违约,则卖方需向买方最多支付
        \[1000000 \times 0.05 \times 2 = 100000,\]
        故B公司最多亏损10万美元。
        \item 若C未违约,则买方需向卖方支付
        \[1000000 \times 0.05 \times 2 = 100000,\]
        故B公司盈利10万美元。
    \end{enumerate}
    \item 假设一家银行A以10\%的固定利率向X公司贷款1亿美元,该银行可以通过与B签订一份总收益互换来对冲。在这份总收益互换中,银行承诺换出这笔贷款的利息加上贷款市场价值的变动部分之和,获得相当于LIBOR$+50$bp的收益。如果现在的LIBOR为9\%,并且一年后贷款的价值从1亿美元跌至9500万美元,分析B的收益情况。\\
    \sol\\
    B要进行两项支付与一项获利:
    \begin{enumerate}[label=(\arabic*)]
        \item 向银行A支付$100,000,000 \times (9 + 0.5)\% = 9,500,000$美元;
        \item 贷款价值下降的支付$100,000,000 - 95,000,000 = 5,000,000$美元;
        \item 固定利率带来的获利$100,000,000 \times 10\% = 10,000,000$美元。
    \end{enumerate}
    故B亏损450万美元。
    \item X银行在市场上筹集到一笔1000万元资金,期限1年,年利率5\%。\\
    假设1:X银行将筹集的1000万元转贷给Y企业,期限1年,贷款利率6\%。\\
    假设2:X银行决定与Z银行做一笔信用违约互换,期望转移这笔贷款的信用风险,须支付0.5\%的买入保护费用。\\
    在假设1的情况中,根据有关规定,X银行必须保证最低8\%的资本金,企业对这笔贷款的风险权重是100\%。在假设2的情况中,即购买了信用违约互换,X银行与Z银行的债权风险权重为20\%。在这两种假设下,分别计算X银行的回报率。\\
    \sol\\
    假设1:X银行的年利润为$8\% \times (1000 \times 1 \times 6\% - 1000 \times 1 \times 5\%) = 0.8$万元,回报率为$\displaystyle \frac{0.8}{1000} = 0.08\%$;\\
    假设2:X银行的年利润为$1000 \times 1 \times 6\% \times 20\% - 1000 \times 0.5\% = 7$万元,回报率为$\displaystyle \frac{7}{1000} = 0.7\%$。
    \item 假设参考资产为100个1年期债券,总面值为100亿元,票面利率均为8.0\%,信用评级为BBB。该债务抵押证券发行A、B、C三个层次的1年期债券,A、B为付息债券,期末一次还本付息,C为零息债券。A、B层分别为70亿元和20亿元,对应的票面利率分别为6.0\%和7.5\%,信用评级分别为AAA和A。C层为股本层,未评级。A层的本息优先于B层,股本层仅在A、B层的本息偿还完毕后才能得到偿付。股本层的金额为10亿元,在全部参考资产均不违约的情况下,收益率为23\%。
    \begin{center}
        \setlength{\tabcolsep}{4mm}{
            \begin{tabular}{c|c|c|c|c|c|c}
                \hline
                \multicolumn{3}{c|}{资产} & \multicolumn{4}{|c}{负债} \\ \hline
                数量/亿元 & 平均利率/\% & 信用评级 & 层次 & 数量/亿元 & 收益率/\% & 信用评级 \\ \hline
                \multirow{3}*{100} & \multirow{3}*{8} & \multirow{3}*{BBB} & A层 & 70 & 6.0 & AAA \\ \cline{4-7}
                ~ & ~ & ~ & B层 & 20 & 7.5 & A \\ \cline{4-7}
                ~ & ~ & ~ & 股本层 & 10 &23.0 & 未评级 \\ \hline
            \end{tabular}}
    \end{center}
    \begin{enumerate}[label=(\arabic*)]
        \item 假设资产只在到期日违约,违约导致5\%的损失和5000万元费用。
        \item 假设违约导致15\%的损失和5000万元费用。
    \end{enumerate}
    在这两种情况下,分析各分券层的损失和收益情况。\\
    \sol
    \begin{enumerate}[label=(\arabic*)]
        \item A层完全不受损失,期末获得70亿元本金和4.2亿元利息,收益率6\%;\\
        B层完全不受损失,期末获得20亿元本金和1.5亿元利息,收益率7.5\%;\\
        股本层得到$100 \times (1-5\%) + 100 \times 8\% - 0.5 - 70 - 4.2 - 20 - 1.5 = 6.8$亿元,收益率为$\displaystyle \frac{6.8-10}{10}=-32\%$。
        \item A层完全不受损失,期末获得70亿元本金和4.2亿元利息,收益率6\%;\\
        B层仅能得到$100 \times (1-15\%) + 100 \times 8\% - 0.5 - 70 - 4.2 = 18.3$亿元,收益率为$\displaystyle \frac{18.3-20}{20} = -8.5 \%$;\\
        股本层全部损失,收益率$-100 \%$。
    \end{enumerate}
    \item CDS合同的一方如果违约,分析合同买方和卖方的损益情况。\\
    \sol\\
    损益情况见下表:
    \begin{center}
        \setlength{\tabcolsep}{2mm}{
        \begin{tabular}{c|c}
            \hline
            买方违约,风险加价上升,CDS合同卖方有利得 & 卖方违约,风险加价上升,CDS 合同买方有利损 \\ \hline
            买方违约,风险加价下降,CDS合同卖方有利损 & 卖方违约,风险加价下降,CDS合同买方有利得 \\ \hline
        \end{tabular}}
    \end{center}
    \item 查资料,分析一份我国的信用衍生品合约。\\
    \omitted
\end{enumerate}