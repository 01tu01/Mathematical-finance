\section{Ross套利定价模型}
\begin{enumerate}
    \item 套利定价模型在形式上与因子模型是否一致?\\
    \sol\\
    套利定价模型在形式上与因子模型是理论一致的,套利定价模型是将因子模型与无套利条件相结合从而得到期望收益和风险之间的关系的模型。
    \item 请比较套利定价模型和CAPM模型,两个模型得出的共同结论是什么?\\
    \sol\\
    比较:
    \begin{enumerate}[label=(\arabic*)]
        \item 模型的假定条件不同\\
        APT假定证券收益率的产生同某些共同因子有关,但这些共同因子到底是什么以及有多少个,模型并没有事先人为地加以指定,而CAPM事先假定证券收益率同市场证券组合的收益率有关。此外,CAPM(无论是简化的CAPM还是扩展的CAPM)的一个基本假定是投资者都以期望收益率和标准差作为分析基础,并按照收益-风险准则选择投资方案,而APT无此假定。
        \item 建立模型的出发点不同\\
        APT考察的是当市场不存在无风险套利而达到均衡时,资产如何均衡定价。而CAPM考察的是当所有投资者都以相似的方法投资,市场最终调节到均衡时,资产如何定价。
        \item 描述形成均衡状态的机理不同\\
        当市场面临证券定价不合理而产生价格压力时,按照APT的思想,即使是少数几个投资者的套利行为也会使市场尽快地重新恢复均衡。而按CAPM的思想,所有投资者都将改变其投资策略,调整他们选择的投资组合,他们共同行为的结果才促使市场重新回到均衡状态。
        \item 风险的来源不同\\
        在CAPM中,证券的风险只与市场组合的$\beta$相关,它只给出了市场风险大小,而没有表明风险来自何处。APT承认有多种因素影响证券价格,从而扩大了资产定价的思考范围(CAPM认为资产定价仅有一个因素),也为识别证券风险的来源提供了分析工具。
        \item 定价范围及精度不同\\
        CAPM是从它的假定条件经逻辑推理得到的,它提供了关于所有证券及证券组合的期望收益率-风险关系的明确描述,只要模型条件满足,以此确定的任何证券或证券组合的均衡价格都是准确的。而APT是从不存在无风险套利的角度推出的,由于市场中有可能存在少数证券定价不合理而整个市场处于均衡的状态(证券数少到不足以产生无风险套利),所以APT提供的均衡定价关系有可能对少数证券不成立。换言之,在满足APT的条件的情况下,用APT的证券或证券组合确定均衡价格,对少数证券的定价可能出现偏差。
    \end{enumerate}
    共同结论:资本市场的均衡关系是固定的。
    \item 假设你用一个双因子模型估计股票$z$的回报率,表达式如下:
    \[r_z=0.5+0.8R_m+0.2R_L+e_z\]其中,$R_m$是市场指数的回报率,$L$代表未料到的流动性的变化。
    \begin{enumerate}[label=(\arabic*)]
        \item 如果市场回报率是10\%,未料到的流动性的变化是3\%,那么股票$z$的回报率是多少?
        \item 如果$L$不变,$R_m$下降5\%,那么股票$z$的回报率将怎样变化?
    \end{enumerate}
    \sol
    \begin{enumerate}[label=(\arabic*)]
        \item 由题知:$R_m = 10\%, R_L = 3\%$,则
        \[r_z=0.5 + 0.8 \times 0.1 + 0.2 \times 0.03 = 58.3\%,\]
        故股票$z$的回报率是58.3\%。
        \item 由题知:$R_m = -5\%, R_L = 0$,则
        \[r_z=0.5 + 0.8 \times (-0.05) + 0.2 \times 0 = 46\%,\]
        故股票$z$的回报率下降4\%。
    \end{enumerate}
    \item 假定市场可以用下面的三种系统风险及相应的风险溢价进行描述。
    \begin{center}
        \setlength{\tabcolsep}{18mm}{
        \begin{tabular}{c|c|c}
            \hline
            风险因素 & $\beta$值 & 风险因素价格\textcolor{red}{(\%)} \\ \hline
            宏观因素$F_1$ & 0.5 & 6 \\ \hline
            宏观因素$F_2$ & 0.3 & 8 \\ \hline
            宏观因素$F_3$ & 1.2 & 3 \\ \hline
        \end{tabular}}
    \end{center}
    问:\begin{enumerate}[label=(\arabic*)]
        \item 如果无风险收益率为3\%,则合理定价下该股票的期望收益率为多少?
        \item 假定三种宏观因素的市场预测值分别为5\%,3\%和2\%,而实际值是4\%,6\%和0,则该股票修正后的收益率为多少?
    \end{enumerate}
    \sol
    \begin{enumerate}[label=(\arabic*)]
        \item $E(R_p) = 3\% + 0.5 \times 6\% + 0.3 \times 8\% + 1.2 \times 3\% = 12\%$,故该股票的期望收益率为12\%。
        \item 各个因素的预测值与实际值不相等,这些非预期变化对资产收益率的影响为:
        \[0.5 \times (4\% - 5\%) + 0.3 \times (6\% - 3\%) + 1.2 \times (0 - 2\%)= -2\%,\]
        则修正后的收益率为$12\%-2\%=10\%$。
    \end{enumerate}
    \item 考虑一个双因子模型,因子$a$和$b$对应的风险溢价分别为4\%和6\%。股票$A$在因子$a$上的$\beta$为1.2,在因子$b$上的$\beta$为0.9。股票$A$的预期收益率是16\%。如果不存在套利机会,无风险收益率是多少?\\
    \sol\\
    设无风险收益率为$i$,由题意:$E(R_p) = i + 1.2 F_1 + 0.9 F_2 + \varepsilon$,则
    \[16\% = i + 1.2 \times 4\% + 0.9 \times 6\% \Rightarrow i = 5.8\%,\]
    故无风险收益率是5.8\%。
    \item 考虑单因子套利定价模型,由三个证券组成的充分分散的资产组合的有关数据见下表。
    \begin{center}
        \setlength{\tabcolsep}{21mm}{
        \begin{tabular}{c|c|c}
            \hline
            证券 & 预期收益率/\% & $\beta$系数 \\ \hline
            $A$ & 10 & 1 \\ \hline
            $B$ & 9 & 2/3 \\ \hline
            $C$ & 4 & 0 \\ \hline
        \end{tabular}}
    \end{center}
    根据以上数据,该资产组合是否存在套利机会?投资者应该如何制定套利策略?\\
    \sol\\
    投资者使用这3种证券构成一个无套利投资组合的条件如下:
    \begin{enumerate}[label=(\arabic*)]
        \item 零投资。分别以$x_1, x_2, x_3$表示构成的无套利组合中各个证券的权重,则应满足如下方程:
        \[x_1+x_2+x_3=0\]
        \item 无风险。分别以$\beta_1,\beta_2,\beta_3$表示各个证券对某个因子的敏感度,投资组合对该因素的敏感度等于各证券敏感度的加权平均。
        \[x_1\beta_1+x_2\beta_2+x_3\beta_3=0\]
        \item 零收益。
        \[x_1E(r_1) + x_2 E(r_2) + x_3 E(r_3) = 0\]
    \end{enumerate}
    符合上述三个条件的投资组合为无套利组合。如果仅符合前两个条件,而组合的投资收益大于0,则为套利组合。不妨设$x_1 = t$,解由前两个方程构成的方程组
    \[\begin{cases}
        t + x_2 + x_3 = 0,\\
        \displaystyle t + \frac{2}{3}x_2 = 0
    \end{cases}\]
    解得$x_2 = -1.5t, x_3 = 0.5t$,代入第三个条件,则
    \[x_1E(r_1) + x_2 E(r_2) + x_3 E(r_3) = -1.23t\]
    则只需$t<0$,就可以构造套利组合。不妨设$x_1 = t = -0.1$,则$x_2 = 0.15, x_3 = -0.05$,即存在套利组合$(-0.1,0.15,-0.05)$,期望收益率为12.3\%。比如购买150元证券$B$,卖出100元证券$A$、50元证券$C$,可以获利123元。
\end{enumerate}
\clearpage