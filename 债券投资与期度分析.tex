\section{债券投资与期度分析}
\begin{enumerate}
    \item 假设市场利率为10\%,一个剩余期限还有18个月、息率为7\%的债券(每年付息两次,刚付完半年的利息),求该债券价格。18个月期的零息债券的到期收益率为多少?\\
    \sol\\
    由题知:$i = 10\%, M = 1.5 \text{年}, C=7\%$,则
    \[p = 100\times \frac{7\%}{10\%} \times \left[1 - \left(1 + \frac{10\%}{2}\right)^{-3}\right]+100\times \left(1+\frac{10\%}{2}\right)^{-3}=95.92,\]
    故该债券价格为95.92元。\\
    并且$N = 3\text{期}$,则
    \[95.92 = 100 \times \frac{7\%}{r/2} \left[1 - \left(1 + \frac{r}{2}\right)^{-3}\right] + 100 \times \left(1 + \frac{r}{2}\right)^{-3} \Rightarrow r = 17.20\%,\]
    则该债券的到期收益率为17.20\%。
    \item 一个2年期债券的息率为8\%,每年付息两次,债券的当前现金价格为103元,那么债券的收益率是多少?\\
    \sol\\
    由题知:$N = 4 \text{期}, C=8\%$,则
    \[103 = 100 \times \frac{8\%}{r/2} \left[1 - \left(1 + \frac{r}{2}\right)^{-4}\right] + 100 \left(1 + \frac{r}{2}\right)^{-4} \Rightarrow r = 14.22\%,\]
    则收益率是7.11\%。
    \item 假设一个投资者在2013年1月1日购买了一个5年期息率为8\%的债券,当时市场利率为6\%。该债券的面值为100000元,到期日为2015年12月31日。计算投资者购买之前的价格,假设债券1年付息一次。\\
    \sol\\
    由题意:$i = 6\%, 2M = 2\text{年}, C=8\%, V= 100000$,则
    \[P = \frac{8\% \times 100000}{6\%} \times \left[1 - \left(1 + 6\%\right)^{-2}\right]+100000\times \left(1+6\%\right)^{-2}=340999.17,\]
    故投资者购买之前的价格为340999.17元。
    \item 假设一个债券的收益率为10\%,计算一年期面值为1000元的零息债券的价格;2年期面值为1000元的零息债券的价格;3年期面值为10000元的零息债券的价格。\\
    \sol\\
    若该债券1年付息一次,则
    \begin{align*}
        & p_1 = 1000(1 + 10\%)^{-1} = 909.09,\\
        & p_2 = 1000(1 + 10\%)^{-2} = 826.45,\\
        & p_3 = 10000(1 + 10\%)^{-3} = 7513.15.
    \end{align*}
    故零息债券的价格分别为909.09元、826.45元、7513.15元。\\
    若该债券半年付息一次,则
    \begin{align*}
        & p_1 = 1000(1 + 10\%/2)^{-2} = 907.03,\\
        & p_2 = 1000(1 + 10\%/2)^{-4} = 822.70,\\
        & p_3 = 10000(1 + 10\%/2)^{-6} = 7462.15.
    \end{align*}
    故零息债券的价格分别为907.03元、822.70元、7462.15元。
    \item 一个年收益率为10\%的3年期债券,其息率为10\%(每年支付一次),面值为10000元,计算该债券的价格。\\
    \sol
    \[p = 10000 \times \frac{10\%}{10\%} [1 - (1 + 10\%)^{-3}] + 10000(1 + 10\%)^{-3} = 10000,\]
    故该债券的价格为10000元。
    \item 一个年收益率为12\%的4年期债券,其息率为8\%(每年支付一次),面值为10000元,计算该债券的期度。\\
    \sol\\
    先计算$P(i)$,则
    \begin{align*}
        P(i) & = \frac{800}{2i} \left[1-\left(1 + i\right)^{-4}\right] + 10000(1 + i)^{-4}\\
        & = \frac{400}{i}\left[1-\left(1 + i\right)^{-4}\right] + 10000(1 + i)^{-4},\\
        \frac{\mathrm{d}\, P(i)}{\mathrm{d} \, i} & = \frac{1600}{i(1+i)^5} - \frac{40000}{(1 + i)^5} + \frac{400\left(\frac{1}{(1 + i)^4} - 1\right)}{i^2}.
    \end{align*}
    期度为
    \begin{align*}
        D(8\%) & = \left.-\frac{\mathrm{d}\, P(i)}{\mathrm{d} \, i}\right| _{i = 8\%} \cdot \frac{1 + 8\%}{P(8\%)} \\
        & = 3.76
    \end{align*}
    故该债券的期度为3.76年。
    \item 假设你接下来两年需要在年末支付每年11700元的学费。目前市场利率为7\%。计算这笔款项的现值和其期度。\\
    \sol\\
    显然这笔款项是零息的,则
    \[p = 11700(1 + 7\%)^{-2} = 10219.23,\]
    故这笔款项的现值为10219.23元。计算$P(i)$,则
    \begin{align*}
        P(i) & = 11700(1 + i)^{-2} = \frac{11700}{(1 + i)^2},\\
        \frac{\mathrm{d}\, P(i)}{\mathrm{d} \, i} & = -\frac{23400}{(1 + i)^3}.
    \end{align*}
    期度为
    \begin{align*}
        D(7\%) & = \left.-\frac{\mathrm{d}\, P(i)}{\mathrm{d} \, i}\right| _{i = 7\%} \cdot \frac{1 + 7\%}{P(7\%)} \\
        & = 2
    \end{align*}
    这笔款项的期度为2年。
    \item 计算以下四种债券的价格和期度。哪个债券对于市场利率的变动最为敏感?
    \begin{center}
        \setlength{\tabcolsep}{6mm}{
        \begin{tabular}{c|c|c|c|c}
            \hline
            证券 & 息率(半年付息一次)/\% & 面值/元 & 期限/年 & 市场利率(年利率)/\% \\ \hline
            A & 10 & 1000 & 1.5 & 15 \\ \hline
            B & 5 & 1000 & 1.5 & 15 \\ \hline
            C & 0 & 1000 & 1.5 & 15 \\ \hline
            D & 0 & 1000 & 1 & 15 \\ \hline
        \end{tabular}}
    \end{center}
    \sol\\
    证券A:
    \begin{align*}
        P_A(i) & = \frac{100}{i} \left[1 - \left(1 + \frac{i}{2}\right)^{-3}\right] + 1000\left(1 + \frac{i}{2}\right)^{-3},\\
        P_A(15\%) & = 934.99,\\
        \frac{\mathrm{d}\, P_A(i)}{\mathrm{d} \, i} & = \frac{100 \left(\frac{1}{{\left(\frac{i}{2} + 1\right)}^3} - 1\right)}{i^2} - \frac{1500}{{\left(\frac{i}{2} + 1\right)}^4} + \frac{150}{i {\left(\frac{i}{2} + 1\right)}^4},\\
        D_A(i) & = \frac{\left(\frac{i}{2} + 1\right) \left(\frac{100 \left(\frac{1}{{\left(\frac{i}{2} + 1\right)}^3} - 1\right)}{i^2} - \frac{1500}{{\left(\frac{i}{2} + 1\right)}^4} + \frac{150}{i {\left(\frac{i}{2} + 1\right)}^4}\right)}{\frac{100 \left(\frac{1}{{\left(\frac{i}{2} + 1\right)}^3} - 1\right)}{i} - \frac{1000}{{\left(\frac{i}{2} + 1\right)}^3}},\\
        D_A(15\%) & = 1.42712.
    \end{align*}
    故证券A的价格为934.99元,期度为1.42712年。\\
    证券B:
    \begin{align*}
        P_B(i) & = \frac{50}{i} \left[1 - \left(1 + \frac{i}{2}\right)^{-3}\right] + 1000\left(1 + \frac{i}{2}\right)^{-3},\\
        P_B(15\%) & = 869.97,\\
        \frac{\mathrm{d}\, P_B(i)}{\mathrm{d} \, i} & = \frac{100 \left(\frac{1}{{\left(\frac{i}{2} + 1\right)}^3} - 1\right)}{i^2} - \frac{1500}{{\left(\frac{i}{2} + 1\right)}^4} + \frac{150}{i {\left(\frac{i}{2} + 1\right)}^4},\\
        D_B(i) & = \frac{\left(\frac{i}{2} + 1\right) \left(\frac{50 \left(\frac{1}{{\left(\frac{i}{2} + 1\right)}^3} - 1\right)}{i^2} - \frac{1500}{{\left(\frac{i}{2} + 1\right)}^4} + \frac{75}{i {\left(\frac{i}{2} + 1\right)}^4}\right)}{\frac{50 \left(\frac{1}{{\left(\frac{i}{2} + 1\right)}^3} - 1\right)}{i} - \frac{1000}{{\left(\frac{i}{2} + 1\right)}^3}},\\
        D_B(15\%) & = 1.46084.
    \end{align*}
    故证券B的价格为869.97元,期度为1.46084年。\\
    证券C:
    \begin{align*}
        P_C(i) & = 1000\left(1 + \frac{i}{2}\right)^{-3},\\
        P_C(15\%) & = 804.96,\\
        \frac{\mathrm{d}\, P_C(i)}{\mathrm{d} \, i} & = \frac{100 \left(\frac{1}{{\left(\frac{i}{2} + 1\right)}^3} - 1\right)}{i^2} - \frac{1500}{{\left(\frac{i}{2} + 1\right)}^4} + \frac{150}{i {\left(\frac{i}{2} + 1\right)}^4},\\
        D_C(i) & = 1.5,\\
        D_C(15\%) & = 1.5.
    \end{align*}
    故证券C的价格为804.96元,期度为1.5年。\\
    证券D:
    \begin{align*}
        P_D(i) & = 1000\left(1 + \frac{i}{2}\right)^{-2},\\
        P_D(15\%) & = 865.33,\\
        \frac{\mathrm{d}\, P_D(i)}{\mathrm{d} \, i} & = \frac{100 \left(\frac{1}{{\left(\frac{i}{2} + 1\right)}^3} - 1\right)}{i^2} - \frac{1500}{{\left(\frac{i}{2} + 1\right)}^4} + \frac{150}{i {\left(\frac{i}{2} + 1\right)}^4},\\
        D_D(i) & = 1,\\
        D_D(15\%) & = 1.
    \end{align*}
    故证券D的价格为865.33元,期度为1年。\\
    考虑变动时,我们不妨将市场利率改变为15.1\%,则
    \begin{align*}
        D_A(15.1\%) & = 1.42706,\\D_B(15.1\%) & = 1.46081,\\D_C(15.1\%) & = 1.5, \\ D_D(15.1\%) & = 1.
    \end{align*}
    显然,证券A的变动幅度最大,债券A对于市场利率的变动最为敏感。
    \item 考虑两个债券组合。资产组合A由一个本金为2000元的1年期零息债券和1个面值为6000元的10年期零息债券组成。资产组合B由一个面值为5000元的5.95年期的零息债券组成。每个债券的收益率为10\%。证明这两个资产组合具有相同的期度。\\
    \pro\\
    先计算资产组合A的期度:
    \begin{align*}
        P_{A1}(i) & = 2000(1 + i)^{-1},\\
        P_{A2}(i) & = 2000(1 + i)^{-10},\\
        D_{A1}(10\%) & = \left.-\frac{\mathrm{d}\, P_{A1}(i)}{\mathrm{d} \, i}\right| _{i = 10\%} \cdot \frac{1 + 10\%}{P_{A1}(10\%)} = 1,\\
        D_{A2}(10\%) & = \left.-\frac{\mathrm{d}\, P_{A2}(i)}{\mathrm{d} \, i}\right| _{i = 10\%} \cdot \frac{1 + 10\%}{P_{A2}(10\%)} = 10,\\
        D_A(10\%) & = \frac{1 \times P_{A1}(10\%) + 10 \times P_{A2}(10\%)}{P_{A1}(10\%) + P_{A2}(10\%)} = 6.
    \end{align*}
    资产组合B的期度:
    \[D_B(10\%) =  \left.-\frac{\mathrm{d}\, P_{B}(i)}{\mathrm{d} \, i}\right| _{i = 10\%} \cdot \frac{1 + 10\%}{P_B(10\%)} = 6.\]
    故这两个资产组合具有相同的期度。
\end{enumerate}
\clearpage