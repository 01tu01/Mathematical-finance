\section{远期与期货}
\begin{enumerate}
    \item 金融期货的定价方法是什么?\\
    \sol\\
    现货-远期平价定理,即现货-期货平价定理。
    \item 举例说明股指期货的定价。\\
    \sol\\
    参见书本161页。
    \item 金融期货套期保值的基本原理何在?\\
    \sol\\
    某一特定商品的期货价格和现货价格受相同经济因素的影响和制约,价格走势是一致的。
    \item 套期保值的种类有哪些?\\
    \sol\\
    套期保值可分为多头套期保值(买入套期保值)和空头套期保值(卖出套期保值)。
    \item 简述买入套期保值和卖出套期保值的适用对象及范围。\\
    \sol\\
    买入套期保值的适用对象及范围:准备在将来某一时间内必须购进某种商品时价格仍能维持在目前认可的水平的商品者。一般用于商品者担心实际买入现货商品时价格上涨。\\
    卖出套期保值的适用对象及范围:准备在未来某一时间内在现货市场上售出实物商品的生产经营者。一般用于生产经营者担心实际卖出现货商品时价格下跌。
    \item 在持有股票组合时,投资者如何利用股指期货进行套期保值?\\
    \sol\\
    如果要保值的股票组合与指数组合不同,我们采用交叉套期保值。当要保值的资产价值与所用的期货合约的标的资产的变化不是完全同步时,要考察两者价格变化的相关关系,并确定合适的套期保值比率。
    \item 分析下面案例:某经销商将在3个月后购买1000吨大豆。在3个月内每吨大豆的价格变化的标准方差为0.056。公司选择购买豆粕期货合约的方法来进行套期保值。在3个月内豆粕期货价格变化的标准方差为0.080,且3个月内大豆价格的变化与3个月内豆粕期货价格变化之间的相关系数为0.9。求最佳的套期比率和经销商应购买的期货合约数目。\textcolor{red}{(题中的标准方差指的是标准差)}\\
    \sol\\
    由题知:$\sigma_s = 0.056, \sigma_f = 0.08, \rho = 0.9$,则
    \[h = \frac{\rho \sigma_s}{\sigma_f} = \frac{0.9 \times 0.056}{0.08} = 0.63,\]
    故最佳的套期比率为0.63,经销商应购买的期货合约数目为1份。
    \item 考虑下面例子中货币期货的套期保值:某年3月1日,美国一家进口商与瑞士一家出口商签订了一份进口2000只瑞士表的合同,约定于3个月后交货付款,每只手表的价格为380瑞士法郎。当时即期汇率为每瑞士法郎0.6309美元。如按这一汇率计算,美国进口商用479484美元即可购得所需的760000瑞士法郎。为避免这一因汇率的不确定变动而可能造成的损失,美国进口商便决定从IMM(芝加哥国际货币市场)买进同年6月份到期的瑞士法郎期货合约,以实施外汇期货的多头套期保值,过程如下,试分析这一过程:
    \begin{center}
        % \setlength{\tabcolsep}{4mm}{
        \begin{tabular}{c|c|c}
            \hline
            日期 & 现货市场 & 期货市场 \\ \hline
            \tabincell{c}{3月\\1日} & \tabincell{l}{签订进口合约,约定3个月后付款760000瑞士法\\郎,按当时即期汇率(1瑞士法郎$=0.6309$美元)\\计算,应支付479484美元。} & \tabincell{l}{买进6月份到期的瑞士法郎期货合约\\6份,成交期货汇率为0.6450,合约总\\价值为483750美元。} \\ \hline
            \tabincell{c}{6月\\1日} & \tabincell{l}{即期汇率升至1瑞士法郎$=0.6540$美元,按此汇率\\购买760000瑞士法郎,共支付497040美元。} & \tabincell{l}{卖出6月份到期的瑞士法郎期货合约\\6份,成交期货汇率为0.6683,合约总\\价值为501225美元。} \\ \hline
            损益 & & \\ \hline
            结果 & \multicolumn{2}{|c}{} \\ \hline
        \end{tabular}
        % }
    \end{center}
    \sol
    \begin{center}
        % \setlength{\tabcolsep}{4mm}{
        \begin{tabular}{c|c|c}
            \hline
            日期 & 现货市场 & 期货市场 \\ \hline
            \tabincell{c}{3月\\1日} & \tabincell{l}{签订进口合约,约定3个月后付款760000瑞士法\\郎,按当时即期汇率(1瑞士法郎$=0.6309$美元)\\计算,应支付479484美元。} & \tabincell{l}{买进6月份到期的瑞士法郎期货合约\\6份,成交期货汇率为0.6450,合约总\\价值为483750美元。} \\ \hline
            \tabincell{c}{6月\\1日} & \tabincell{l}{即期汇率升至1瑞士法郎$=0.6540$美元,按此汇率\\购买760000瑞士法郎,共支付497040美元。} & \tabincell{l}{卖出6月份到期的瑞士法郎期货合约\\6份,成交期货汇率为0.6683,合约总\\价值为501225美元。} \\ \hline
            损益 & \tabincell{l}{比预期多支付$497040 - 479484 = 17556$美元,即亏\\损了17556美元。} & 盈利$501225 - 483750 = 17475$美元。 \\ \hline
            结果 & \multicolumn{2}{|l}{收获$-17556 + 17475 = -81$美元,亏损18美元,投资者没有避免利率上升的损失。} \\ \hline
        \end{tabular}
        % }
    \end{center}
\end{enumerate}
\clearpage