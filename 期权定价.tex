\section{期权定价}
\begin{enumerate}
    \item 解释出售一个看涨期权与购买一个看跌期权的区别。\\
    \sol\\
    看涨期权:它给予期权的持有者在给定时间或在此时间之前的任一时刻按规定的价格买入一定数量某种资产的权利,但不负有必须买进的义务。\\
    看跌期权:它给予其持有者在给定时间或在此时间之前的任一时刻按规定的价格卖出一定数量某种资产的权利,但不负有必须卖出的义务。
    \item 从金融期权交易的基本策略来看,期权购买者与期权出售者的盈亏有何特点?\\
    \sol\\
    期权购买者的收益随市场价格的变化而波动,是不固定的,其亏损则只限于购买期权的保险费;期权出售者的收益只是出售期权的保险费,其亏损则是不固定的。期货的交易双方则都面临着无限的盈利和无止境的亏损。
    \item 风险中性定价机制的内容是什么?\\
    \sol\\
    在市场不存在任何套利可能性的条件下,如果衍生证券的价格依然依赖于可交易的基础证券,那么这个衍生证券的价格是与投资者的风险态度无关的。
    \item 投资者相信股票价格将有巨大变动但方向不确定。请说明投资者能采用哪些不同策略,并解释它们之间的不同点。\\
    \sol\\
    投资者能采用以下六种策略:宽跨式期权(Strangle)、跨式期权(Straddle)、落式期权(Strip)、吊式期权(Strap)、倒置日历价差期权及倒置蝶式价差期权。\\
    当股价有巨大变动,这些策略都能有正的利润。宽跨式期权较跨式期权便宜,但要求股价有更大变动以确保正利润;落式期权及吊式期权都比跨式期权贵,当股价有大幅度下降,落式期权将有更大利润,当股价有大幅度上升,吊式期权将有更大利润。宽跨式期权、跨式期权、落式期权及吊式期权的利润都随股价变动范围增大而增加。相比而言,倒置价差期权存在潜在利润而无论股价波动幅度多大。
    \item 什么是保护性的看跌期权?看涨期权的什么头寸等价于有保护性的看跌期权?\\
    \sol\\
    有保护的看跌期权由看跌期权多头与标的资产多头组成,由期权平价公式可知,其等价于看涨期权多头与一笔固定收入的组合。
    \item 多头同价对敲和空头同价对敲分别适用于何种场合?如何操作?\\
    \sol\\
    多头同价对敲:当投资者预期股票价格有较大波动时。同时买入标的股票、施权价、到期日都完全相同的看涨期权和看跌期权。\\
    空头同价对敲:当投资者预期股票价格没有较大被动时。同时卖出标的股票、施权价、到期日都完全相同的看涨期权和看跌期权。
    \item 一个看涨期权的$\delta$值为0.8意味着什么?\\
    \sol\\
    如果某看涨期权之标的物的市场价格上涨1元,则该期权的期权费上涨0.8元,是平价看涨期权。
    \item 某年3月,投资者预期在两个月后可取得一笔资金,总额为500000元。他对A公司股票看好,所以他计划在收到这笔资金后即全部投资于A公司股票。假定当时A公司股票的市场价格为25元/股,则该投资者预期收到的500000元资金可购买A公司股票20000股。但是,他担心A公司股票在未来的两个月内将有较大幅度的上涨,从而使他失去由股价上涨而产生的收益。为此,他决定以A公司股票的看涨期权作套期保值。其具体的操作是购买以A公司股票为标的物的看涨期权200个。这种期权的敲定价格为25元/股(即平价期权);弃权费为1元/股,所以200个期权的期权费总额为20000元,期限为两个月,期权样式为欧式。在两个月后,A公司股票的市场价格可能有如下三种不同情况:
    \begin{enumerate}[label=(\arabic*)]
        \item 市场价格不变,即仍然为25元/股。
        \item 市场价格下跌,如跌至20元/股。
        \item 市场价格果真大幅上涨,如涨至35元/股。
    \end{enumerate}
    分析这三种情况下,投资者的收益情况。如果投资者在两个月后未能如期收到该笔资金,而A公司的股票价格已经上涨,他又应该怎么操作?\\
    \sol\\
    先计算1个看涨期权有$x$股股票,则
    \[200x \times 25 = 20000 \Rightarrow x = 4,\]
    则该投资者一共购买了800股股票,则
    \begin{enumerate}[label=(\arabic*)]
        \item 投资人收益为
        \[800 \times 25 - 20000 - 800 \times 1 = -800\text{元},\]
        故投资者亏损800元,还持有499200元。
        \item 投资人收益为
        \[800 \times 20 - 20000 - 800 \times 1 = -4800\text{元},\]
        故投资者亏损4800元,还持有495200元。
        \item 投资人收益为
        \[800 \times 35 - 20000 - 800 \times 1 = 7200\text{元},\]
        故投资者盈利7200元,还持有507200元。
    \end{enumerate}
    如果投资者在两个月后未能如期收到该笔资金,则考虑两种情况,原行为与新行为。\\
    原行为:
    \[-500000 + 20000 \times 35 = 200000\text{元}.\]
    新行为:
    \[-20000 + 800\times 35 -800 \times 1 = 7200\text{元}.\]
    故如果投资者在两个月后未能如期收到该笔资金,投资者应该采取原行为,即直接购买A公司股票20000股。
    \item 假设当前股票价格为19元,对于该股票的欧式看涨期权和看跌期权的执行价格均为20元,期限为3个月,这两个期权价格均为3元,目前无风险年利率为10\%,问是否存在套利机会?\\
    \sol\\
    由题知:$\displaystyle S_0 = 19, K = 20, P_c = P_p = 3, T = \frac{1}{4}\text{年}, r = 10\%$,则
    \begin{align*}
        P_c + K(1 + r)^{-T} & = 22.53,\\
        P_p + S_0 & = 22 < 22.53.
    \end{align*}
    由看涨期权和零息债券(存款)构成的投资组合在今天的价值22.53高于由看跌期权和股票构成的投资组合在今天的价值22。因此我们可以通过购入价值低的证券组合同时卖出价值高的证券组合来制造套利机会。
    \item 考虑一个有效期为6个月、执行价格为40元的欧式看涨期权。已知无风险年利率为10\%,波动率为20\%,当前股票价格为42元。该看涨期权的价格应该是多少?具有相同有效期和执行价格的欧式看跌期权的价格又为多少呢?\\
    \sol\\
    由题知:$\tau = 0.5, K = 40, r = 10\%, \sigma = 20\%, S_t = 42$,则
    \begin{align*}
        d_1 & = \frac{\ln \left(\frac{42}{40}\right) + \left(0.1 + \frac{0.2^2}{2}\right) \times 0.5}{0.2 \times \sqrt{0.5}} = 0.769,\\
        d_2 & = d_1 - 0.2 \times \sqrt{0.5} = 0.628,\\
        N(d_1) & = 0.7791,\\
        N(d_2) & = 0.7350,\\
        C_t & = S_tN(d_1) - K \mathrm{e}^{-r \tau}N(d_2)\\
        & = 42 \times 0.7791 - 40\mathrm{e}^{-0.1 \times 0.5} \times 0.7350\\
        & = 4.76.
    \end{align*}
    故看涨期权的价格是4.76元。
    \begin{align*}
        P_t & = K\mathrm{e}^{-r(T-t)}N(-d_2)-S_tN(-d_1)\\
        & = 40\mathrm{e}^{-0.1 \times 0.5} \times 0.2650 - 42 \times 0.2209\\
        & = 0.81.
    \end{align*}
    故看跌期权的价格为0.81元。
\end{enumerate}
\clearpage